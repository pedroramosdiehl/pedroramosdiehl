\documentclass[10pt, letterpaper]{article}

% Packages:
\usepackage[
    ignoreheadfoot, % set margins without considering header and footer
    top=2 cm, % seperation between body and page edge from the top
    bottom=2 cm, % seperation between body and page edge from the bottom
    left=2 cm, % seperation between body and page edge from the left
    right=2 cm, % seperation between body and page edge from the right
    footskip=1.0 cm, % seperation between body and footer
    % showframe % for debugging
]{geometry} % for adjusting page geometry
\usepackage{titlesec} % for customizing section titles
\usepackage{tabularx} % for making tables with fixed width columns
\usepackage{array} % tabularx requires this
\usepackage[dvipsnames]{xcolor} % for coloring text
\definecolor{primaryColor}{RGB}{0, 0, 0} % define primary color
\usepackage{enumitem} % for customizing lists
\usepackage{fontawesome5} % for using icons
\usepackage{amsmath} % for math
\usepackage[
    pdftitle={Pedro Diehl's CV},
    pdfauthor={Pedro Diehl},
    pdfcreator={LaTeX with RenderCV},
    colorlinks=true,
    urlcolor=primaryColor
]{hyperref} % for links, metadata and bookmarks
\usepackage[pscoord]{eso-pic} % for floating text on the page
\usepackage{calc} % for calculating lengths
\usepackage{bookmark} % for bookmarks
\usepackage{lastpage} % for getting the total number of pages
\usepackage{changepage} % for one column entries (adjustwidth environment)
\usepackage{paracol} % for two and three column entries
\usepackage{ifthen} % for conditional statements
\usepackage{needspace} % for avoiding page brake right after the section title
\usepackage{iftex} % check if engine is pdflatex, xetex or luatex

% Ensure that generate pdf is machine readable/ATS parsable:
\ifPDFTeX
    \input{glyphtounicode}
    \pdfgentounicode=1
    \usepackage[T1]{fontenc}
    \usepackage[utf8]{inputenc}
    \usepackage{lmodern}
\fi

\usepackage{charter}

% Some settings:
\raggedright
\AtBeginEnvironment{adjustwidth}{\partopsep0pt} % remove space before adjustwidth environment
\pagestyle{empty} % no header or footer
\setcounter{secnumdepth}{0} % no section numbering
\setlength{\parindent}{0pt} % no indentation
\setlength{\topskip}{0pt} % no top skip
\setlength{\columnsep}{0.15cm} % set column seperation
\pagenumbering{gobble} % no page numbering

\titleformat{\section}{\needspace{4\baselineskip}\bfseries\large}{}{0pt}{}[\vspace{1pt}\titlerule]

\titlespacing{\section}{
    % left space:
    -1pt
}{
    % top space:
    0.3 cm
}{
    % bottom space:
    0.2 cm
} % section title spacing

\renewcommand\labelitemi{$\vcenter{\hbox{\small$\bullet$}}$} % custom bullet points
\newenvironment{highlights}{
    \begin{itemize}[
        topsep=0.10 cm,
        parsep=0.10 cm,
        partopsep=0pt,
        itemsep=0pt,
        leftmargin=0 cm + 10pt
    ]
}{
    \end{itemize}
} % new environment for highlights


\newenvironment{highlightsforbulletentries}{
    \begin{itemize}[
        topsep=0.10 cm,
        parsep=0.10 cm,
        partopsep=0pt,
        itemsep=0pt,
        leftmargin=10pt
    ]
}{
    \end{itemize}
} % new environment for highlights for bullet entries

\newenvironment{onecolentry}{
    \begin{adjustwidth}{
        0 cm + 0.00001 cm
    }{
        0 cm + 0.00001 cm
    }
}{
    \end{adjustwidth}
} % new environment for one column entries

\newenvironment{twocolentry}[2][]{
    \onecolentry
    \def\secondColumn{#2}
    \setcolumnwidth{\fill, 4.5 cm}
    \begin{paracol}{2}
}{
    \switchcolumn \raggedleft \secondColumn
    \end{paracol}
    \endonecolentry
} % new environment for two column entries

\newenvironment{threecolentry}[3][]{
    \onecolentry
    \def\thirdColumn{#3}
    \setcolumnwidth{, \fill, 4.5 cm}
    \begin{paracol}{3}
    {\raggedright #2} \switchcolumn
}{
    \switchcolumn \raggedleft \thirdColumn
    \end{paracol}
    \endonecolentry
} % new environment for three column entries

\newenvironment{header}{
    \setlength{\topsep}{0pt}\par\kern\topsep\centering\linespread{1.5}
}{
    \par\kern\topsep
} % new environment for the header

\newcommand{\placelastupdatedtext}{% \placetextbox{<horizontal pos>}{<vertical pos>}{<stuff>}
  \AddToShipoutPictureFG*{% Add <stuff> to current page foreground
    \put(
        \LenToUnit{\paperwidth-2 cm-0 cm+0.05cm},
        \LenToUnit{\paperheight-1.0 cm}
    ){\vtop{{\null}\makebox[0pt][c]{
        \small\color{gray}\textit{Last updated in September 2024}\hspace{\widthof{Last updated in September 2024}}
    }}}%
  }%
}%

% save the original href command in a new command:
\let\hrefWithoutArrow\href

% new command for external links:


\begin{document}
    \newcommand{\AND}{\unskip
        \cleaders\copy\ANDbox\hskip\wd\ANDbox
        \ignorespaces
    }
    \newsavebox\ANDbox
    \sbox\ANDbox{$|$}

    \begin{header}
        \fontsize{25 pt}{25 pt}\selectfont Pedro Ramos Krauze Diehl

        \vspace{5 pt}

        \normalsize
        \mbox{Goiânia, Goiás, Brasil}%
        \kern 5.0 pt%
        \AND%
        \kern 5.0 pt%
        \mbox{\hrefWithoutArrow{mailto:pedroramosdiehl@gmail.com}{pedroramosdiehl@gmail.com}}%
        \kern 5.0 pt%
        \AND%
        \kern 5.0 pt%
        \mbox{\hrefWithoutArrow{tel:+5562982188015}{BR (62) 982188015}}%
        \kern 5.0 pt%
        \AND%
        \kern 5.0 pt%
        \mbox{\hrefWithoutArrow{tel:+33-07-49-49-50-36}{FR 07 49 49 50 36}}%
        % \kern 5.0 pt%
        % \AND%
        % \kern 5.0 pt%
        % \mbox{\hrefWithoutArrow{https://yourwebsite.com/}{yourwebsite.com}}%
        \kern 5.0 pt%
        \AND%
        \kern 5.0 pt%
        \mbox{\hrefWithoutArrow{https://linkedin.com/in/pedro-ramos-krauze-diehl-5670a011a}{linkedin.com/in/pedro-ramos-krauze-diehl-5670a011a}}%
        \kern 5.0 pt%
        \AND%
        \kern 5.0 pt%
        \mbox{\hrefWithoutArrow{https://github.com/pedroramosdiehl}{github.com/pedroramosdiehl}}%
    \end{header}

    \vspace{5 pt - 0.3 cm}

    \section{Education}

        \begin{twocolentry}{
            January 2016 – March 2023
        }
        \textbf{Universidade Federal de Goiás}, Bachelor of Computer Engineering\end{twocolentry}

        \vspace{0.10 cm}
        \begin{onecolentry}
            \begin{highlights}
                \item Final Project: \textbf{Sistema de Presença em Sala de Aula com Reconhecimento Facial por meio da Visão Computacional}.
                \begin{highlightsforbulletentries}
                    \item Authors: Daniel Moraes dos Santos and Pedro Ramos Krauze Diehl
                    \item Advisor: Prof. Dr. Adriano César Santana
                    \item Description: Developed a classroom attendance system utilizing facial recognition through computer vision techniques.
                \end{highlightsforbulletentries}
                \item \textbf{Key Coursework:} Computer Architecture, Machine Learning, Distributed Systems
            \end{highlights}
        \end{onecolentry}


    \section{Experience}

        \begin{twocolentry}{
            February 2024 – Present
        }
            \textbf{Arquiteto de Soluções}, Centro de Excelência em IA (CEIA) \& Centro de Competência EMBRAPII (AKCIT)
        \end{twocolentry}
        \vspace{0.10 cm}
        \begin{onecolentry}
            \begin{highlights}
                \item Designed and implemented R\&D projects focusing on AI and immersive technologies.
                \item Developed scalable solutions for data-driven applications.
            \end{highlights}
        \end{onecolentry}

        \vspace{0.2 cm}

        \begin{twocolentry}{
            May 2023 – Present
        }
            \textbf{Arquiteto de Software}, Universidade Federal de Goiás -- Goiânia, Goiás, Brasil
        \end{twocolentry}
        \vspace{0.10 cm}
        \begin{onecolentry}
            \begin{highlights}
                \item Architected and developed software solutions for internal university systems.
                \item Implemented full-stack applications using Angular, Java, and Docker.
            \end{highlights}
        \end{onecolentry}

        \vspace{0.2 cm}

        \begin{twocolentry}{
            January 2022 – March 2023
        }
            \textbf{Desenvolvedor Pleno}, Universidade Federal de Goiás -- Goiânia, Goiás, Brasil
        \end{twocolentry}
        \vspace{0.10 cm}
        \begin{onecolentry}
            \begin{highlights}
                \item Developed and maintained critical systems including SIG, SISCONCURSO, and SICAD.
                \item Built and deployed the Sempre UFG mobile application, enhancing student engagement.
            \end{highlights}
        \end{onecolentry}

        \vspace{0.2 cm}

        \begin{twocolentry}{
            January 2019 – December 2020
        }
            \textbf{Bolsista em Desenvolvimento Mobile}, Universidade Federal de Goiás
        \end{twocolentry}
        \vspace{0.10 cm}
        \begin{onecolentry}
            \begin{highlights}
                \item Developed mobile applications such as CS Candidato, RU UFG, and Minha UFG using Ionic and Spring Boot.
                \item Maintained PHP-based systems and enhanced platform reliability.
            \end{highlights}
        \end{onecolentry}

        \vspace{0.2 cm}

    \section{Skills}
        \begin{onecolentry}
            \textbf{Programming Languages:} Java, JavaScript, TypeScript, Python, PHP
        \end{onecolentry}

        \vspace{0.10 cm}
        \begin{onecolentry}
            \textbf{Frontend Development:} Angular, Ionic, HTML5, CSS3
        \end{onecolentry}

        \vspace{0.10 cm}
        \begin{onecolentry}
            \textbf{Backend Development:} Spring Boot, Django, Flask, Python
        \end{onecolentry}

        \vspace{0.10 cm}
        \begin{onecolentry}
            \textbf{Cloud & DevOps:} AWS, Kubernetes, Docker
        \end{onecolentry}

        \vspace{0.10 cm}
        \begin{onecolentry}
            \textbf{Database Management:} PostgreSQL, MySQL, MongoDB, Oracle
        \end{onecolentry}

        \vspace{0.10 cm}
        \begin{onecolentry}
            \textbf{Infrastructure & Tools:} Linux, Git, Grafana, Nginx, Proxmox
        \end{onecolentry}

    \section{Projects}

            \textbf{Sempre UFG Mobile Application}
        \end{twocolentry}
        \vspace{0.10 cm}
        \begin{onecolentry}
            \begin{highlights}
                \item Developed and deployed a mobile app for student access to university services.
                \item Tools: Ionic, Angular, Spring Boot.
            \end{highlights}
        \end{onecolentry}

        \vspace{0.2 cm}

            \textbf{RU UFG Mobile Application}
        \end{twocolentry}
        \vspace{0.10 cm}
        \begin{onecolentry}
            \begin{highlights}
                \item Created an application to streamline university dining services.
                \item Tools: Ionic, Angular, AWS.
            \end{highlights}
        \end{onecolentry}

\end{document}